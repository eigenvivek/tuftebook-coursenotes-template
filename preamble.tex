% Packages
\usepackage{parskip}
\usepackage{epsfig}
\usepackage{amsfonts}
\usepackage{amssymb}
\usepackage{amstext}
\usepackage{amsmath}
\usepackage{xspace}
\usepackage{hyperref}
\usepackage{amsthm}
\usepackage{natbib}
\usepackage{color}
\usepackage{cancel}
\usepackage{bbm}
\usepackage{optidef}
\allowdisplaybreaks

% Environments
\theoremstyle{plain}
\newtheorem{theorem}{Theorem}[chapter]
\newtheorem{cor}{Corollary}[theorem]
\newtheorem{lemma}[theorem]{Lemma}
\newtheorem{prop}{Proposition}[chapter]
\newtheorem{obs}{Observation}[chapter]

\theoremstyle{definition}
\newtheorem{defn}{Definition}[chapter]

\theoremstyle{remark}
\newtheorem{example}{Example}[chapter]
\newtheorem{exercise}{Exercise}[chapter]
\newtheorem{remark}{Remark}[chapter]
\newtheorem{question}{Question}
\newtheorem{answer}{Answer}

% Statistics
\renewcommand{\P}{\ensuremath{\mathbb{P}}}
\renewcommand{\Pr}[1]{\ensuremath{\mathbb{P} \left( #1 \right) }}
\newcommand{\E}[1]{\ensuremath{\mathbb{E} \left[ #1 \right] }}
\newcommand{\var}[1]{\ensuremath{\mathrm{Var} \left( #1 \right) }}
\newcommand{\mse}[1]{\ensuremath{\mathrm{MSE} \left( #1 \right) }}
\newcommand{\bias}[1]{\ensuremath{\mathrm{Bias} \left( #1 \right) }}

\newcommand{\1}[1]{\ensuremath{\mathbbm 1_{\{#1\}}}}

\newcommand{\Y}{\ensuremath{\mathbf Y}}
\newcommand{\y}{\ensuremath{\mathbf y}}

\newcommand{\iid}{\ensuremath{\stackrel{\mathrm{iid}}{\sim}}}
\newcommand{\ind}{\perp \!\!\! \perp}

\newcommand{\bern}{\mathrm{Bernoulli}}
\newcommand{\pois}{\mathrm{Poisson}}
\newcommand{\mult}{\mathrm{Multinomial}}

% Fields
\newcommand{\R}{\ensuremath{\mathbb R}}
\newcommand{\C}{\ensuremath{\mathbb C}}
\newcommand{\N}{\ensuremath{\mathbb N}}
\newcommand{\Q}{\ensuremath{\mathbb Q}}
\newcommand{\F}{\ensuremath{\mathbb F}}
\newcommand{\K}{\ensuremath{\mathbb K}}
\newcommand{\Z}{\ensuremath{\mathbb Z}}
\newcommand{\B}{\ensuremath{\mathcal B}}
\renewcommand{\H}{\ensuremath{\mathcal H}}

% Useful things
\newcommand{\G}{\ensuremath{\mathcal G}}
\newcommand{\e}{\epsilon}
\newcommand{\sse}{\subseteq}
\newcommand{\union}{\cup}
\newcommand{\ra}{\rightarrow}
\newcommand{\ceil}[1]{\ensuremath{\left\lceil#1\right\rceil}}
\newcommand{\floor}[1]{\ensuremath{\left\lfloor#1\right\rfloor}}
\newcommand{\ip}[2]{\left\langle #1, #2\right\rangle}
\DeclareMathOperator*{\argmax}{arg\,max\ }
\DeclareMathOperator*{\argmin}{arg\,min\ }
\newcommand{\rr}{\stackrel{RR}{\sim}}
\newcommand{\set}[2]{ \left\{ #1 \;:\; #2 \right\} }
\newcommand{\T}{\ensuremath{\mathbf{T}}}

% Vectors and matrices
\newcommand{\zero}{\vec{0}}
\newcommand{\ones}{\vec{1}}
\newcommand{\norm}[1]{\lVert#1\rVert}

% Operators
\DeclareMathOperator{\vspan}{span}
\DeclareMathOperator{\im}{\mathrm{im}}
\DeclareMathOperator{\rank}{\mathrm{rank}}
\DeclareMathOperator{\RREF}{\mathrm{RREF}}
\DeclareMathOperator{\trace}{\mathrm{trace}}
\DeclareMathOperator{\diag}{\mathrm{diag}}

% Misc symbols
\DeclareRobustCommand\iff{\;\Longleftrightarrow\;}
\DeclareRobustCommand\implies{\;\Longrightarrow\;}
\DeclareRobustCommand\limplies{\;\Longleftarrow\;}
\DeclareRobustCommand{\contradiction}{\ensuremath{{\Rightarrow\mspace{-2mu}\Leftarrow}}}

%%% Book formatting

%% Reformat the title
\makeatletter
\renewcommand{\maketitlepage}{
    \begingroup
    \setlength{\parindent}{0pt}
    {\fontsize{24}{24}\selectfont\textit{\@author}\par}
    \vspace{1.75in}{\fontsize{36}{54}\selectfont\@title\par}
    \vspace{0.5in}{\fontsize{14}{14}\selectfont\textsf{\smallcaps{\@date}}\par}
    \vfill{\fontsize{14}{14}\selectfont\textit{\@publisher}\par}
    \thispagestyle{empty}
    \endgroup
}
\makeatother

%% Add numbers to the chapters
\setcounter{secnumdepth}{2}

\titleformat{\chapter}
    {\huge\rmfamily\itshape}
    {Lecture \thechapter.}
    {8pt}
    {}
    []

%% Reformat table of contents
\usepackage{titlesec}
\usepackage{titletoc}

\renewcommand{\chaptername}{Lecture}

\titlecontents{part}
    [0pt]
    {\addvspace{0.25\baselineskip}}
    {\allcaps{Part~\thecontentslabel}}
    {\allcaps{Part~\thecontentslabel}}
    {}
    [\vspace*{0.5\baselineskip}]
    
\titlecontents*{chapter}
    [4em]
    {}
    {\chaptername~ \thecontentslabel\quad}
    {}
    {\hfill\contentspage}
    [\vspace*{0.5\baselineskip}]